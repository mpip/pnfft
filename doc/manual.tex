\PassOptionsToPackage{pdfpagelabels=true}{hyperref}
\documentclass[a4paper,11pt,final,openright,headsepline]{scrreprt}
% parskip: no indentation of paragraphs, german style
%\usepackage{layout} % show layout boarders

\usepackage[english]{babel} % wordbreaks
\usepackage[utf8]{inputenc}
\usepackage[T1]{fontenc}
\usepackage[fixlanguage]{babelbib} % easily change bib language

%% Vector fonts for PDF
% \usepackage{ae} % deprecated package, use lmodern instead
\usepackage{lmodern} %standard latex font

%\usepackage{makeidx} % automatic index generation, required for nomencl.sty
%\usepackage{nomencl} % important symbols in a table at the beginning of document

%% AMSMath-packages
\usepackage{amsmath}
\usepackage{amsthm}
\usepackage{amssymb}
% \usepackage{amsrefs}
% \usepackage{textcmds}
\usepackage{exscale} % Correct font scaling in formulas

% \usepackage{subfig}
\usepackage{graphicx} % Graphics for figures
\usepackage[svgnames, table, hyperref]{xcolor} % 
\usepackage{paralist} % compact itemize, enumerate, ...
\usepackage{listings} % source code in LaTeX
\usepackage{multirow} % combine multiple rows in arrays
\usepackage{rotating}

%% improvements of LaTeX enviroments
\usepackage{scrhack} % avoid warning of scrreprt when loading float package
\usepackage{float}
% \usepackage{verbatim}
% \usepackage{array}

% \usepackage{url} % provides \url command for bibtex
\usepackage{hyperref} % provides \url command for bibtex and links to jump within documents
\hypersetup{plainpages=false, colorlinks, linkcolor=black, citecolor=black, urlcolor=blue,
pdftitle={PNFFT Users Manual},
pdfauthor={Michael Pippig}, pdfstartview={FitBH}}

%% adjust numbering
\numberwithin{figure}{chapter}
\numberwithin{table}{chapter}
\numberwithin{equation}{chapter}

%% activate for algorithms
% \usepackage[chapter]{algorithm}
% \usepackage{algorithmic,algorithmic-fix}
% \floatname{algorithm}{Algorithmus} % use german title for algorithms
% \numberwithin{algorithm}{chapter}


%% activate for algorithms
\usepackage{algpseudocode}
\usepackage[chapter]{algorithm}
\usepackage{algorithmicx}
% \floatname{algorithm}{Algorithmus} % use german title for algorithms
% \numberwithin{algorithm}{chapter}

\usepackage{xspace}
\usepackage{nicefrac}

\usepackage{todonotes}


%% activate for compact page layout
% \usepackage{geometry}
% \geometry{top=30.4mm, left=30.4mm, text={155mm,240mm}, headheight=10mm, headsep=5mm, includemp, marginparwidth=15.4mm}

%% activate for headline with chapter information on every page
\usepackage{scrpage2}
\pagestyle{scrheadings}
\automark[section]{chapter}
\clearscrheadings
\lehead{\pagemark}
\rehead{\leftmark}
\rohead{\pagemark}
\lohead{\rightmark}
\ofoot[]{}
\cfoot[]{}
\ifoot[]{}

\renewcommand*{\thefootnote}{\fnsymbol{footnote}}

%% the quotchap document style redefines the \chapter and \chapter* commands to
%% create fancy chapter head pages with huge chapter numbers (possibly greyed) and
%% provides commands for adding quotations in the upper left corner of these pages.
% \usepackage[grey]{quotchap}


%% very special purpose packages
% \usepackage{faktor} % provides a symbol for factor groups
% \usepackage{slashbox} % diagonaly divide an array field


%% experimental
% \usepackage[color]{showkeys} % show all reference keys
% \definecolor{refkey}{gray}{.75}
% \definecolor{labelkey}{gray}{0.75}
% \usepackage{epstopdf} % include .eps files with pdflatex
% \usepackage{marginnote}
%\usepackage{pgf}
%\usepackage{jkpgf}
%\usepackage{pstricks}
%% make pdf-indexfile for inverse search - needs compatible pdf viewer
% \synctex=1
% \usepackage{pdfsync} % deprecated package, use synctex instead


\hyphenation{equi-spaced non-equi-spaced NFFT FFTW FFT BGFFT nicht-äqui-dis-tant nicht-äqui-dis-tante}

\newtheorem{Satz}{Satz}[chapter] % define Satz enviroment for german docs
\input{shortcuts.tex} % Shortcuts for math symbols


\pagestyle{scrheadings}

\automark[section]{chapter} % auto
\clearscrheadings
\lehead{\pagemark}
\rehead{\leftmark}
\rohead{\pagemark}
\lohead{\rightmark}
\ofoot[]{}
\cfoot[]{}
\ifoot[]{}

% \includeonly{intro} % work on single chapters



\newcommand{\nfftarrow}{\osetarrow{\textsf{NFFT}}}



\begin{document}
  \setcounter{equation}{0}

%   \thispagestyle{empty}
  \begin{titlepage}
    \begin{center}
      \vspace*{3cm}
      {\Huge {\textbf{\textsf{PNFFT Users Manual\\}}}}

      \vspace*{1cm}
      for version \pnfftversion\, \today

      \vspace*{3cm}
      Michael Pippig\\
      Chemnitz University of Technology\\
      Department of Mathematics\\
      09107 Chemnitz, Germany

      \vspace*{3cm}
      Download Parallel Nonequispaced Fast Fourier Transform Software Library at \\
      {\large \url{http://www.tu-chemnitz.de/~mpip/software}}
    \end{center}
    \vfill
    michael.pippig@mathematik.tu-chemnitz.de
  \end{titlepage}

  \tableofcontents


  %%%%%%%%%%%%%%%%%%%%%%%%%%%%%%%%%%%%%%%%%%%%%%%%%%%%%%%%%%%%%%%%%%%%%%%%%%%%%%%
  \chapter{Introduction}\label{kap:intro}
  %%%%%%%%%%%%%%%%%%%%%%%%%%%%%%%%%%%%%%%%%%%%%%%%%%%%%%%%%%%%%%%%%%%%%%%%%%%%%%%
  %%------------------------------------------------------------------------------
\section{Introduction}
%------------------------------------------------------------------------------

\begin{compactitem}
  \item NFFT consists of 3 steps: ...
  \item describe the data decomposition
  \item make clear that 3d-decomposition is possible but 2d-decomposition is more natural
\end{compactitem}
\begin{compactitem}
  \item Details on the algorithms can be found in \cite{PiPo12}.
  \item Details on the underlying parallel FFT can be found in \cite{Pi12}.
\end{compactitem}

local NFFT specific variables
\begin{compactitem}
  \item \verb+ptrdiff_t local_M+
  \item \verb+ptrdiff_t local_N[3]+, \verb+ptrdiff_t local_N_start[3]+
  \item \verb+double lower_border[3]+, \verb+double upper_border[3]+
\end{compactitem}

global NFFT specific variables
\begin{compactitem}
  \item \verb+ptrdiff_t N[3]+
  \item \verb+ptrdiff_t n[3]+
  \item \verb+int m+
  \item \verb+double x_max[3]+
\end{compactitem}



2d data decomposition with non-transposed Fourier coefficients
\begin{equation*}
  \hat N_0 / P_0 \times \hat N_1 / P_1 \times \hat N_2
  \nfftarrow
  C_0 / P_0 \times C_1 / P_1 \times C_2
\end{equation*}

2d data decomposition with transposed Fourier coefficients
\begin{equation*}
  \hat N_1 / P_0 \times \hat N_2 / P_1 \times \hat N_0
  \nfftarrow
  C_0 / P_0 \times C_1 / P_1 \times C_2
\end{equation*}

3d data decomposition with non-transposed Fourier coefficients
\begin{equation*}
  \hat N_0 / P_0 \times \hat N_1 / P_1 \times \hat N_2 / P_2
  \nfftarrow
  C_0 / P_0 \times C_1 / P_1 \times C_2 / P_2
\end{equation*}

3d data decomposition with transposed Fourier coefficients with $P_2 = Q_0 Q_1$
\begin{equation*}
  \hat N_1 / (P_0 Q_0) \times \hat N_2 / (P_1 Q_1) \times \hat N_0
  \nfftarrow
  C_0 / P_0 \times C_1 / P_1 \times C_2 / P_2
\end{equation*}


\subsection{Parallel NFFT Workflow}

\begin{compactitem}
  \item create a simple test program and describe it here
\end{compactitem}


\begin{compactitem}
  \item get block distribution of Fourier coefficients and nodes
  \item call PNFFT planner
  \item init Fourier coefficients and nodes
  \item precomputations that depend on the nodes
  \item execute PNFFT plan
  \item read results
  \item finalize PNFFT
\end{compactitem}




%------------------------------------------------------------------------------
\section{Special Features}
%------------------------------------------------------------------------------


\begin{compactitem}
  \item flags \code{PNFFT_OMIT_DECONV}, \code{PNFFT_OMIT_FFT}, \code{PNFFT_OMIT_CONV}
  \item \code{pnfft_trafo} and \code{pnfft_adj} can be called with \code{nodes == NULL} if and only if \code{PNFFT_OMIT_CONV} is set. 
    Direct computation with \code{nodes == NULL} does not make sence. Therefore, it immediately returns without doing anything.
    Passing only some nodes as NULL pointer may lead to deadlocks. I.e., in parallel calls always pass all or none nodes as NULL.
  \item \code{PNFFT_COMPUTE_F}, \code{PNFFT_COMPUTE_GRAD_F}, \code{PNFFT_HESSIAN_F} must be set in planning step in order to compute the correct FFT-values.
        This makes a difference in the number of FFTs for ik-diff but not for ad-diff. However, also for ad-diff we need the info in order to optimize precomputation
        of fct. value and derivatives (TODO: do we always precompute fct. values, 1st order and 2nd derivative, or do we check the flags?)
        They can also be changed on node-level?
  \item \code{PNFFT_COMPUTE_ACCUMULATED} with pnfft_adj makes that the Fourier coefficients are not zeroed at the beginning.
    I.e., multiple calls of adjoint interpolation accumulate in the Fourier coefficients. Typical use case is interpolation for multiple sets of nodes (charges/dipoles),
    without immediate computation of FFT (use \code{PNFFT_OMIT_FFT} and \code{PNFFT_OMIT_DECONV}). After several convolutions we compute the FFT and deconvolution
    on the accumulated values.
  \item \code{PNFFT_COMPUTE_ACCUMULATED} with pnfft_trafo makes that the fct. values, gradients and Hessian are not zeroed at the beginning of trafo.
    Use case: compute deconv and FFT in one call. Compute fct. values, gradients, Hessian for same set of nodes in separate calls. TODO: Does this flag make sense with trafo?
  \item Question: Is \code{PNFFT_OMIT_CONV} equivalent to not set any of these flags \code{PNFFT_COMPUTE_F}, \code{PNFFT_COMPUTE_GRAD_F}, \code{PNFFT_COMPUTE_HEASSIAN_F}?
    Answer: No! These 3 flags are also needed for the correct FFT execution, e.g. with ik-diff we need to call/skip FFTs for each derivative.
    But: Is this part of planning or execution? Part of both. 
      Part of planning because of memory allocation. 
      Part of execution because FFTs are expensive and we only call them if necesarry.
\end{compactitem}


\subsection{Transposed Fourier coefficients}
A parallel transpose FFT algorithm typically ends up with a transposed order of the output array.
Start with


Similar to PFFT, our parallel NFFT supports an optimization flag that disables the backward transpositions. Therefore, one must work on a transposed array of Fourier coefficients.


\subsection{Truncated Torus}
PNFFT support the special case, where the nodes $\mathbf x_j$ fulfill the restriction $\mathbf x_j \in \left[-\frac{C_0}{2},\frac{C_0}{2} \right]$

Copy from PNFFT paper:\\
In addition, we pay special attention to the case where all the nonequispaced nodes $\mathbf x_j$
are contained in a special subset of the torus $\T^3$. For $\mathbf C=(C_0,C_1,C_2)^\top\in\R^3$ with $0<C_0,C_1,C_2\le 1$
we define the truncated torus $\T^3_{\mathbf C} := [-\frac{C_0}{2},\frac{C_0}{2})\times [-\frac{C_1}{2},\frac{C_1}{2}) \times [-\frac{C_2}{2},\frac{C_2}{2})$.
For the parallel NFFT we assume $\mathbf x_j\in\T^3_{\mathbf C}$ for every $j=1,\hdots,M$.
Obviously, for $C_0=C_1=C_2=1$ this corresponds to the serial NFFT, where the nodes $\mathbf x_j$ are contained in the whole three-dimensional torus $\T^3$.
This slight generalization is necessary in order to assure a load balanced distribution of nodes $\mathbf x_j$ whenever the nodes are
concentrated in the center of the box.


%------------------------------------------------------------------------------
\section{Download and Install}
%------------------------------------------------------------------------------
\begin{compactitem}
  \item download, configure, make
  \item Advice for developers: install new version of autotools (add the script here)
\end{compactitem}





 % ohne Seitenumbrueche
  %------------------------------------------------------------------------------
\section{Introduction}
%------------------------------------------------------------------------------

\begin{compactitem}
  \item NFFT consists of 3 steps: ...
  \item describe the data decomposition
  \item make clear that 3d-decomposition is possible but 2d-decomposition is more natural
\end{compactitem}
\begin{compactitem}
  \item Details on the algorithms can be found in \cite{PiPo12}.
  \item Details on the underlying parallel FFT can be found in \cite{Pi12}.
\end{compactitem}

local NFFT specific variables
\begin{compactitem}
  \item \verb+ptrdiff_t local_M+
  \item \verb+ptrdiff_t local_N[3]+, \verb+ptrdiff_t local_N_start[3]+
  \item \verb+double lower_border[3]+, \verb+double upper_border[3]+
\end{compactitem}

global NFFT specific variables
\begin{compactitem}
  \item \verb+ptrdiff_t N[3]+
  \item \verb+ptrdiff_t n[3]+
  \item \verb+int m+
  \item \verb+double x_max[3]+
\end{compactitem}



2d data decomposition with non-transposed Fourier coefficients
\begin{equation*}
  \hat N_0 / P_0 \times \hat N_1 / P_1 \times \hat N_2
  \nfftarrow
  C_0 / P_0 \times C_1 / P_1 \times C_2
\end{equation*}

2d data decomposition with transposed Fourier coefficients
\begin{equation*}
  \hat N_1 / P_0 \times \hat N_2 / P_1 \times \hat N_0
  \nfftarrow
  C_0 / P_0 \times C_1 / P_1 \times C_2
\end{equation*}

3d data decomposition with non-transposed Fourier coefficients
\begin{equation*}
  \hat N_0 / P_0 \times \hat N_1 / P_1 \times \hat N_2 / P_2
  \nfftarrow
  C_0 / P_0 \times C_1 / P_1 \times C_2 / P_2
\end{equation*}

3d data decomposition with transposed Fourier coefficients with $P_2 = Q_0 Q_1$
\begin{equation*}
  \hat N_1 / (P_0 Q_0) \times \hat N_2 / (P_1 Q_1) \times \hat N_0
  \nfftarrow
  C_0 / P_0 \times C_1 / P_1 \times C_2 / P_2
\end{equation*}


\subsection{Parallel NFFT Workflow}

\begin{compactitem}
  \item create a simple test program and describe it here
\end{compactitem}


\begin{compactitem}
  \item get block distribution of Fourier coefficients and nodes
  \item call PNFFT planner
  \item init Fourier coefficients and nodes
  \item precomputations that depend on the nodes
  \item execute PNFFT plan
  \item read results
  \item finalize PNFFT
\end{compactitem}




%------------------------------------------------------------------------------
\section{Special Features}
%------------------------------------------------------------------------------


\begin{compactitem}
  \item flags \code{PNFFT_OMIT_DECONV}, \code{PNFFT_OMIT_FFT}, \code{PNFFT_OMIT_CONV}
  \item \code{pnfft_trafo} and \code{pnfft_adj} can be called with \code{nodes == NULL} if and only if \code{PNFFT_OMIT_CONV} is set. 
    Direct computation with \code{nodes == NULL} does not make sence. Therefore, it immediately returns without doing anything.
    Passing only some nodes as NULL pointer may lead to deadlocks. I.e., in parallel calls always pass all or none nodes as NULL.
  \item \code{PNFFT_COMPUTE_F}, \code{PNFFT_COMPUTE_GRAD_F}, \code{PNFFT_HESSIAN_F} must be set in planning step in order to compute the correct FFT-values.
        This makes a difference in the number of FFTs for ik-diff but not for ad-diff. However, also for ad-diff we need the info in order to optimize precomputation
        of fct. value and derivatives (TODO: do we always precompute fct. values, 1st order and 2nd derivative, or do we check the flags?)
        They can also be changed on node-level?
  \item \code{PNFFT_COMPUTE_ACCUMULATED} with pnfft_adj makes that the Fourier coefficients are not zeroed at the beginning.
    I.e., multiple calls of adjoint interpolation accumulate in the Fourier coefficients. Typical use case is interpolation for multiple sets of nodes (charges/dipoles),
    without immediate computation of FFT (use \code{PNFFT_OMIT_FFT} and \code{PNFFT_OMIT_DECONV}). After several convolutions we compute the FFT and deconvolution
    on the accumulated values.
  \item \code{PNFFT_COMPUTE_ACCUMULATED} with pnfft_trafo makes that the fct. values, gradients and Hessian are not zeroed at the beginning of trafo.
    Use case: compute deconv and FFT in one call. Compute fct. values, gradients, Hessian for same set of nodes in separate calls. TODO: Does this flag make sense with trafo?
  \item Question: Is \code{PNFFT_OMIT_CONV} equivalent to not set any of these flags \code{PNFFT_COMPUTE_F}, \code{PNFFT_COMPUTE_GRAD_F}, \code{PNFFT_COMPUTE_HEASSIAN_F}?
    Answer: No! These 3 flags are also needed for the correct FFT execution, e.g. with ik-diff we need to call/skip FFTs for each derivative.
    But: Is this part of planning or execution? Part of both. 
      Part of planning because of memory allocation. 
      Part of execution because FFTs are expensive and we only call them if necesarry.
\end{compactitem}


\subsection{Transposed Fourier coefficients}
A parallel transpose FFT algorithm typically ends up with a transposed order of the output array.
Start with


Similar to PFFT, our parallel NFFT supports an optimization flag that disables the backward transpositions. Therefore, one must work on a transposed array of Fourier coefficients.


\subsection{Truncated Torus}
PNFFT support the special case, where the nodes $\mathbf x_j$ fulfill the restriction $\mathbf x_j \in \left[-\frac{C_0}{2},\frac{C_0}{2} \right]$

Copy from PNFFT paper:\\
In addition, we pay special attention to the case where all the nonequispaced nodes $\mathbf x_j$
are contained in a special subset of the torus $\T^3$. For $\mathbf C=(C_0,C_1,C_2)^\top\in\R^3$ with $0<C_0,C_1,C_2\le 1$
we define the truncated torus $\T^3_{\mathbf C} := [-\frac{C_0}{2},\frac{C_0}{2})\times [-\frac{C_1}{2},\frac{C_1}{2}) \times [-\frac{C_2}{2},\frac{C_2}{2})$.
For the parallel NFFT we assume $\mathbf x_j\in\T^3_{\mathbf C}$ for every $j=1,\hdots,M$.
Obviously, for $C_0=C_1=C_2=1$ this corresponds to the serial NFFT, where the nodes $\mathbf x_j$ are contained in the whole three-dimensional torus $\T^3$.
This slight generalization is necessary in order to assure a load balanced distribution of nodes $\mathbf x_j$ whenever the nodes are
concentrated in the center of the box.


%------------------------------------------------------------------------------
\section{Download and Install}
%------------------------------------------------------------------------------
\begin{compactitem}
  \item download, configure, make
  \item Advice for developers: install new version of autotools (add the script here)
\end{compactitem}





 % mit Seitenumbruechen
  \include{fortran}

%------------------------------------------------------------------------------
\subsubsection*{How to Deal with FFT Shifts in Parallel}
%------------------------------------------------------------------------------
PFFT calculates
\begin{equation*}
  h_l = \sum_{k=0}^{N-1} \hat h_k \eim{kl/n},
  \quad l=0,\hdots,L
\end{equation*}
with $n\ge N$ and $n\ge L$.
Step 2 of our PNFFT algorithm requires the shifted index sets
\begin{equation*}
  g_l = \sum_{k=-N/2}^{N/2-1} \hat g_k \eim{kl/n},
  \quad l=-L/2,\hdots,L/2-1
\end{equation*}
A common technique to deal with this problem is to call FFT shift, i.e., set
\begin{equation*}
  \hat h_k
  =
  \begin{cases}
    g_k,     & k=0,\hdots,N/2-1 \\
    g_{k+N}, & k=N/2,\hdots,N-1
  \end{cases}
\end{equation*}
\begin{equation*}
  g_l
  =
  \begin{cases}
    h_{l+N}, & l=-N/2,\hdots,-1 \\
    h_l,     & l=0,\hdots,N/2-1
  \end{cases}
\end{equation*}
However, because of the parallel data decomposition, this involves explicit data communication.
Instead we apply the FFT shifts in frequency domain.
Use the translation property
\begin{equation*}
  \hat h_k = \hat g_k \eip{kL/2/n}
\end{equation*}
and the modulation property
\begin{equation*}
  h_{l-N/2} = \eim{lN/2/n}
\end{equation*}
of the discrete Fourier transform.

User can choose to shift the input (\verb+PNFFT_SHIFT_INPUT+) and/or to shift the output (\verb+PNFFT_SHIFT_OUTPUT+).

One-dimensional example
\begin{align*}
  f(x)
  &=
    \sum_{k=-\frac{N}{2}}^{\frac{N}{2}-1} \hat f_k \eim{kx} \\
  &\approx
    \sum_{l=-\frac{n}{2}}^{\frac{n}{2}-1}\left( \sum_{k=-\frac{N}{2}}^{\frac{N}{2}-1} \frac{\hat f_k}{\varphi_k} \eim{k\frac{l}{n}} \right) \varphi\left(x-\frac{l}{n}\right) \\
  &=
    \sum_{l=-\frac{n}{2}+\frac{n_o}{2}}^{\frac{n}{2}+\frac{n_o}{2}-1} \left(
      \sum_{k=0}^{N-1} \frac{\hat f_{(k-\frac{N}{2})}}{\varphi_{(k-\frac{N}{2})}} \eim{(k-\frac{N}{2})\frac{(l-\frac{n_o}{2})}{n}}
    \right) \varphi\left( \left(x+\frac{n_o}{2n}\right)-\frac{l}{n}\right) \\
  &=
    \sum_{l=-n/2+n_o/2}^{n/2+n_o/2-1} \left[
      \e^{+\pi\ti N\frac{l}{n}} \sum_{k=0}^{N-1} \left(
        \frac{\hat f_{(k-\frac{N}{2})}}{\hat\varphi_{(k-\frac{N}{2})}}
        \e^{+\pi\ti(k-\frac{N}{2})\frac{n_o}{n}}
      \right)
      \eim{k\frac{l}{n}}
    \right] \varphi\left( x-\frac{l-\frac{n_o}{2}}{n} \right)
\end{align*}
\begin{align*}
  \textrm{lo}
  &=
    \frac{\textrm{local\_no\_start} - \textrm{no/2}}{\textrm{n}} \\
  \textrm{up}
  &=
    \frac{\textrm{local\_no\_start} + \textrm{local\_no} - \textrm{no/2}}{\textrm{n}}
  =
    \textrm{lo} + \frac{\textrm{local\_no}}{\textrm{n}}
\end{align*}



\section{Init plans}

\begin{lstlisting}
  PNFFT_EXTERN PNX(plan) PNX(init_guru)(
        int d,
        const INT *N, const INT *n, const R *x_max,
        INT local_M, int m,
        unsigned pnfft_flags, unsigned fftw_flags,
        MPI_Comm comm_cart);
\end{lstlisting}

\begin{lstlisting}
#define PNFFT_PRE_PHI_HAT           (1U<< 0)
#define PNFFT_PRE_PHI_HUT           ((PRE_PHI_HAT))

#define PNFFT_FG_PSI                (1U<< 1)

#define PNFFT_PRE_CONST_PSI         (1U<< 2)
#define PNFFT_PRE_LIN_PSI           (1U<< 3)
#define PNFFT_PRE_QUAD_PSI          (1U<< 4)
#define PNFFT_PRE_CUB_PSI           (1U<< 5)
#define PNFFT_PRE_INTPOL_PSI        ((PNFFT_PRE_CONST_PSI| PNFFT_PRE_LIN_PSI| PNFFT_PRE_QUAD_PSI| PNFFT_PRE_CUB_PSI))

#define PNFFT_MALLOC_F_HAT          (1U<< 6)
\end{lstlisting}
\begin{lstlisting}
#define PNFFT_FFT_OUT_OF_PLACE      (0U)
#define PNFFT_FFT_IN_PLACE          (1U<< 7)
#define PNFFT_INTERLACED            (1U<< 8)
#define PNFFT_SHIFTED_F_HAT         (1U<< 9)
#define PNFFT_SHIFTED_X             (1U<< 10)
#define PNFFT_TRANSPOSED_NONE       (0U)
#define PNFFT_TRANSPOSED_F_HAT      (1U<< 11)

#define PNFFT_DIFF_IK               (1U<< 12)
#define PNFFT_DIFF_AD               (1U<< 13)

#define PNFFT_WINDOW_KAISER_BESSEL  (0U)
#define PNFFT_WINDOW_GAUSSIAN       (1U<< 14)
#define PNFFT_WINDOW_BSPLINE        (1U<< 15)
#define PNFFT_WINDOW_SINC_POWER     (1U<< 16)
#define PNFFT_WINDOW_BESSEL_I0      (1U<< 17)

#define PNFFT_SORT_NODES            (1U<< 18)
\end{lstlisting}

% #define PNFFT_PRE_ONE_PSI    ((PNFFT_PRE_INTPOL_PSI| PNFFT_PRE_FG_PSI| PNFFT_PRE_PSI| PNFFT_PRE_FULL_PSI))



\begin{lstlisting}
  PNX(nodes) PNX(init_nodes)(
      INT local_M, unsigned pnfft_flags);
\end{lstlisting}
\begin{lstlisting}
#define PNFFT_MALLOC_X         (1U<< 0)
#define PNFFT_MALLOC_F         (1U<< 1)
#define PNFFT_MALLOC_GRAD_F    (1U<< 2)
#define PNFFT_MALLOC_HESSIAN_F (1U<< 3)

#define PNFFT_REAL_F           (1U<< 4)
\end{lstlisting}



\begin{lstlisting}
  void PNX(precompute_psi)(
      PNX(plan) ths, PNX(nodes) nodes, unsigned precompute_flags);
\end{lstlisting}
\begin{lstlisting}
#define PNFFT_PRE_TENSOR      (0U)
#define PNFFT_PRE_FULL        (1U<< 0)
#define PNFFT_PRE_PSI         (1U<< 1)
#define PNFFT_PRE_GRAD_PSI    (1U<< 2)
#define PNFFT_PRE_HESSIAN_PSI (1U<< 3)
\end{lstlisting}
Pre-computation uses the kind of window evaluation that was initialized in the plan, e.g., interpolation from look-up tables, fast Gaussian gridding, or direct evaluation.




\section{Execute plans}
\begin{lstlisting}
void pfft_trafo(pfft_plan plan, pfft_nodes nodes, unsigned compute_flags);
\end{lstlisting}
Hereby, \code{compute_flags} can be
\begin{compactitem}
  \item \code{PNFFT_COMPUTE_DIRECT}
  \item \code{PNFFT_COMPUTE_F}
  \item \code{PNFFT_COMPUTE_GRAD_F}
  \item \code{PNFFT_COMPUTE_HESSIAN_F}
  \item \code{PNFFT_OMIT_DECONV}
  \item \code{PNFFT_OMIT_FFT}
  \item \code{PNFFT_OMIT_CONV}
\end{compactitem}

\begin{lstlisting}
void pfft_adj(pfft_plan plan, pfft_nodes nodes, unsigned compute_flags);
\end{lstlisting}
Hereby, \code{compute_flags} can be the same as above but not \code{PNFFT_COMPUTE_HESSIAN_F}.
\begin{compactitem}
  \item \code{PNFFT_COMPUTE_ACCUMULATED}
\end{compactitem}

\section{Finalize plans}
\begin{lstlisting}
  void PNX(free_nodes)(
      PNX(nodes) ths, unsigned pnfft_finalize_flags);
\end{lstlisting}
\begin{lstlisting}
#define PNFFT_FREE_X           ((PNFFT_MALLOC_X))
#define PNFFT_FREE_F_HAT       ((PNFFT_MALLOC_F_HAT))
#define PNFFT_FREE_F           ((PNFFT_MALLOC_F))
#define PNFFT_FREE_GRAD_F      ((PNFFT_MALLOC_GRAD_F))
#define PNFFT_FREE_HESSIAN_F   ((PNFFT_MALLOC_HESSIAN_F))
#define PNFFT_FREE_ALL         ((PNFFT_FREE_X | PNFFT_FREE_F | PNFFT_FREE_GRAD_F | PNFFT_FREE_HESSIAN_F))
\end{lstlisting}


%%%%%%%%%%%%%%%%%%%%%%%%%%%%%%%%%%%%%%%%%%%%%%%%%%%%%%%%%%%%%%%%%%%%%%%%%%%%%%%
% \selectbiblanguage{english}
\bibliographystyle{bababbrv-fl}
% \bibliographystyle{abbrv}
\bibliography{bibliography}
%%%%%%%%%%%%%%%%%%%%%%%%%%%%%%%%%%%%%%%%%%%%%%%%%%%%%%%%%%%%%%%%%%%%%%%%%%%%%%%

\end{document}